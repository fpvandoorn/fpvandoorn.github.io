%----------------------------------------------------------------------------------------
%	PACKAGES AND THEMES
%----------------------------------------------------------------------------------------

\documentclass[xcolor=table]{beamer}
\mode<presentation> {
\usetheme{Boadilla}
\usecolortheme{seahorse}
}

\usepackage{pgfpages}

\usepackage{array}
\usepackage[T1]{fontenc}
\usepackage[utf8x]{inputenc}
\usepackage{graphicx,bussproofs,tikz,multirow,wasysym}
\usepackage{tikz,mathdots}
\usetikzlibrary{arrows,calc,positioning,backgrounds}
\beamertemplatenavigationsymbolsempty
%\setbeameroption{show notes}
%% \setbeamertemplate{note page}[plain]

\usepackage[utf8x]{inputenc}
\usepackage{amssymb}
\definecolor{keywordcolor}{rgb}{0.7, 0.1, 0.1}   % red
\definecolor{tacticcolor}{rgb}{0.1, 0.2, 0.6}    % blue
\definecolor{commentcolor}{rgb}{0.4, 0.4, 0.4}   % grey
\definecolor{symbolcolor}{rgb}{0.0, 0.1, 0.6}    % blue
\definecolor{sortcolor}{rgb}{0.1, 0.5, 0.1}      % green
\definecolor{attributecolor}{rgb}{0.7, 0.1, 0.1} % red

\usepackage{listings}
\def\lstlanguagefiles{lstlean.tex}
%% \lstset{language=lean}
\lstset{language=lean,breakatwhitespace}


\newcommand{\sy}{\ensuremath{^{-1}}}
\newcommand{\bool}{\ensuremath{\mathbf{2}}}
\newcommand{\fa}[2]{\ensuremath{\Pi(#1),\ #2}}
\newcommand{\fax}[2]{\ensuremath{\Pi#1,\ #2}}
\newcommand{\ex}[2]{\ensuremath{\Sigma(#1),\ #2}}
\newcommand{\exx}[2]{\ensuremath{\Sigma#1,\ #2}}
\newcommand{\N}{\mathbb{N}}
\newcommand{\eps}{\ensuremath{\varepsilon}}
\definecolor{darkgreen}{RGB}{0,100,0}
\definecolor{grey}{RGB}{80,80,80}
\newcommand{\na}{\mbox{}\\}
\newcommand{\nb}{\mbox{}\\[2pt]}
\newcommand{\nc}{\mbox{}\\[6pt]}
\newcommand{\nn}{\mbox{}\\ \mbox{}\\}
\newcommand{\base}{\textnormal{base}}
\newcommand{\colimit}{\textnormal{colimit}}
\newcommand{\lp}{\textnormal{loop}}
\newcommand{\refl}{\textnormal{refl}}
%% \newcommand{\red}[1]{\alert{#1}}
%% \newcommand{\blue}[1]{\textcolor{blue}{#1}}
\newcommand{\green}[1]{\textcolor{darkgreen}{#1}}
%% \newcommand{\grey}[1]{\textcolor{grey}{#1}}
\renewcommand\arraystretch{1.3}
\DeclareMathOperator{\type}{Type}
\DeclareMathOperator{\UU}{\mathcal{U}}
\newcommand{\formalizedtable}{%
\begin{tabular}{c|ccccccccccccccc}
           & 1 & 2 & 3 & 4 & 5 & 6 & 7 & 8 & 9 & 10 & 11 & 12 & 13 & 14 & 15 \\ \hline
\textbf{1} &   &   &   &   & + & + & + & + & + &   & + & + &   &   &   \\
\textbf{2} & + & + & + & + &   & + & + & + & + & + & + & + & + & + & + \\
\textbf{3} & + & + & + & + & ½ & + & + & + & + &   & + &   &   &   &   \\
\textbf{4} & - & + & + & + &   & + & + & + & + &   &   &   &   &   &   \\
\textbf{5} & - &   & ½ & - & - &   &   & ½ &   &   &   &   &   &   &   \\
\textbf{6} &   & + & + & + & + & + & + & + & ¾ & ¼ & ¾ & + &   &   &   \\
\textbf{7} & + & + & + & - & ¾ & - & - &   &   &   &   &   &   &   &   \\
\textbf{8} & + & + & + & + & + & ¾ & + & + & + & ¼ &   &   &   &   &   \\
\textbf{9} & ¾ & + & + & ½ & ¾ & ½ & - & - & - &   &   &   &   &   &   \\
\textbf{10}& ¼ & - & - & - & - &   &   &   &   &   &   &   &   &   &   \\
\textbf{11}& - & - & - & - & - & - &   &   &   &   &   &   &   &   &   \\
\end{tabular}}

%\usepackage{diagrams}
%\diagramstyle[labelstyle=\scriptstyle]


  \tikzset{
    invisible/.style={opacity=0},
    visible on/.style={alt={#1{}{invisible}}},
    alt/.code args={<#1>#2#3}{%
      \alt<#1>{\pgfkeysalso{#2}}{\pgfkeysalso{#3}} % \pgfkeysalso doesn't change the path
    },
  }


\title{Homotopy Type Theory in Lean}

\author{Floris van Doorn}

\institute[CMU]{Department of Philosophy\\
Carnegie Mellon University\nb
\url{leanprover.github.io}}

\date{14 July 2016}

\begin{document}

{\setbeamertemplate{headline}[default]
\begin{frame}
  \titlepage
\end{frame}
}

%% \begin{frame}{Outline}

%% \begin{itemize}
%%  \item The Lean Theorem Prover
%%  \item Lean's kernel
%%  \item Lean's elaborator
%%  \item The HoTT library
%% \end{itemize}

%% \end{frame}

%% \begin{frame}{Formal verification}

%% Formal methods can be used to verify correctness:
%% \begin{itemize}
%% \item verifying that a circuit description, an algorithm, or a network or security protocol meets its specification; or
%% \item verifying that a mathematical statement is true.
%% \end{itemize}

%% \end{frame}

%% \begin{frame}{Formal verification in industry}

%% Formal methods are commonly used:
%% \begin{itemize}
%%  \item Intel and AMD use ITP to verify processors.
%%  \item Microsoft uses formal tools such as Boogie and SLAM to verify programs and drivers.
%%  \item CompCert has verified the correctness of a C compiler.
%%  \item Airbus uses formal methods to verify avionics software.
%%  \item Toyota uses formal methods for hybrid systems to verify control systems.
%%  \item Formal methods were used to verify Paris' driverless line 14 of the Metro.
%%  \item The NSA uses (it seems) formal methods to verify cryptographic algorithms.
%% \end{itemize}

%% \end{frame}

%% \begin{frame}{Formal verification in mathematics}

%% There is no sharp line between industrial and mathematical verification:
%% \begin{itemize}
%%  \item Designs and specifications are expressed in mathematical terms.
%%  \item Claims rely on background mathematical knowledge.
%% \end{itemize}

%% \bigskip

%% Mathematics has a different character, however:
%% \begin{itemize}
%%  \item Problems are conceptually deeper, less heterogeneous.
%%  \item More user interaction is needed.
%% \end{itemize}

%% \end{frame}

%% \begin{frame}{Interactive theorem proving}

%% Working with a proof assistant, users construct a formal axiomatic proof.

%% \bigskip

%% In most systems, this proof object can be extracted and verified independently.

%% \end{frame}

%% \begin{frame}{Interactive theorem proving}

%% Some systems with large mathematical libraries:
%% \begin{itemize}
%%  \item Mizar (set theory)
%%  \item HOL (simple type theory)
%%  \item Isabelle (simple type theory)
%%  \item HOL light (simple type theory)
%%  \item Coq (constructive dependent type theory)
%%  \item ACL2 (primitive recursive arithmetic)
%%  \item PVS (classical dependent type theory)
%% \end{itemize}

%% \end{frame}

%% \begin{frame}{Interactive theorem proving}

%% Some theorems formalized to date:
%% \begin{itemize}
%%  \item the prime number theorem
%%  \item the four-color theorem
%%  \item the Jordan curve theorem
%%  \item G\"odel's first and second incompleteness theorems
%%  \item Dirichlet's theorem on primes in an arithmetic progression
%%  \item Cartan fixed-point theorems
%%  \item the central limit theorem
%% \end{itemize}

%% \bigskip

%% There are good libraries for elementary number theory, real and complex analysis, point-set topology, measure-theoretic probability, abstract algebra, Galois theory, \ldots

%% \end{frame}

%% \begin{frame}{Interactive theorem proving}

%% Georges Gonthier and coworkers verified the  Feit-Thompson Odd Order Theorem in Coq.
%% \begin{itemize}
%%  \item The original 1963 journal publication ran 255 pages.
%%  \item The formal proof is constructive.
%%  \item The development includes libraries for finite group theory, linear algebra, and representation theory.
%% \end{itemize}

%% \bigskip

%% The project was completed on September 20, 2012, with roughly
%% \begin{itemize}
%%  \item 150,000 lines of code,
%%  \item 4,000 definitions, and
%%  \item 13,000 lemmas and theorems.
%% \end{itemize}

%% \end{frame}

%% \begin{frame}{Interactive theorem proving}

%% Hales announced the completion of the formal verification of the Kepler conjecture (\emph{Flyspeck}) in August 2014.

%% \smallskip

%% \begin{itemize}
%%  \item Most of the proof was verified in HOL light.
%%  \item The classification of tame graphs was verified in Isabelle.
%%  \item Verifying several hundred nonlinear inequalities required roughly 5000 processor hours on the Microsoft Azure cloud.
%% \end{itemize}

%% \end{frame}

%% \begin{frame}{Homotopy Type Theory}

%% Homotopy Type Theory opens up new avenues for formal verification.

%% \bigskip

%% It provides a framework for:
%% \begin{itemize}
%%  \item ``synthetic'' reasoning about spaces up to homotopy
%%  \item reasoning about structures of interest in algebraic topology
%%  \item reasoning structures in general, and higher-order equivalences between them
%%  \item reasoning about generic computation and invariance under change of representation
%% \end{itemize}

%% \end{frame}

\begin{frame}{The Lean Theorem Prover}

Lean is a new interactive theorem prover, developed principally by Leonardo de Moura at Microsoft Research, Redmond.

\bigskip

It was ``announced'' in the summer of 2015.

\bigskip

It is open source, released under a permissive license, Apache 2.0.

\bigskip

The goal is to make it a community project, like Clang.

\end{frame}

\begin{frame}{The Lean Theorem Prover}

The aim is to bring interactive and automated reasoning together, and build
\begin{itemize}
 \item an interactive theorem prover with powerful automation
 \item an automated reasoning tool that
 \begin{itemize}
 \item produces (detailed) proofs,
 \item has a rich language,
 \item can be used interactively, and
 \item is built on a verified mathematical library.
 \end{itemize}
\end{itemize}

\end{frame}

\begin{frame}{The Lean Theorem Prover}

Lean is a designed to be a mature system, rather than an experimental one.
\begin{itemize}
 \item Take advantage of existing theory.
 \item Build on strengths of existing interactive and automated theorem provers.
 \item Craft clean but pragmatic solutions.
\end{itemize}

\bigskip

We have drawn ideas and inspiration from Coq, SSReflect, Isabelle, Agda, and Nuprl, among others.

\end{frame}

%% \begin{frame}{The Lean Theorem Prover}

%% Goals:
%% \begin{itemize}
%%  \item Verify hardware, software, and hybrid systems.
%%  \item Verify mathematics.
%%  \item Combine powerful automation with user interaction.
%%  \item Support reasoning and exploration.
%%  \item Support formal methods in education.
%%  \item Create an eminently powerful, usable system.
%%  \item Bring formal methods to the masses.
%% \end{itemize}

%% \end{frame}

\begin{frame}{The Lean Theorem Prover}

Notable features:
\begin{itemize}
 \item based on a powerful dependent type theory
 \item written in C++, with multi-core support
 \item small, trusted kernel with an independent type checker
 \item standard and HoTT instantiations
 \item powerful elaborator
 \item can use proof terms or tactics
 \item Emacs mode with proof-checking on the fly
 \item browser version runs in javascript
 \item already has a respectable library
 \item automation is now the main focus
\end{itemize}

\end{frame}

\begin{frame}{Contributors}

Currently working on the code base: Leonardo de Moura, Daniel Selsam, Lev Nachman, Soonho Kong

\bigskip

Currently working the standard library: Jeremy Avigad, Rob Lewis, Jacob Gross

\bigskip

Currently working on the HoTT library: Floris van Doorn, Ulrik Buchholtz, Jakob von Raumer

\bigskip

Contributors: Cody Roux, Joe Hendrix, Parikshit Khanna, Sebastian Ullrich, Haitao Zhang, Andrew Zipperer, and many others.


\end{frame}

\begin{frame}[fragile]{Lean's kernel}

Lean's kernel implements dependent type theory with
\begin{itemize}
 \item A hierarchy of (non-cumulative) universes:
       \lstinline~Type.{0} : Type.{1} : Type.{2} : Type.{3} : ...~
 \item universe polymorphism:
       \lstinline~definition id.{u} {A : Type.{u}} : A → A := λa, a~
 \item dependent products: \lstinline{Πx : A, B}
 \item inductive types (\`a la Dybjer, constructors and recursors)
\end{itemize}

\bigskip

The kernel is smaller and simpler than those of Coq and Agda.

\bigskip

Daniel Selsam has written an independent type checker in Haskell, which is less than 2,000 lines
long.

\bigskip

The kernel can be instantiated into two modes, a standard mode and the HoTT mode.

\end{frame}

\frame[containsverbatim]{\frametitle{Lean's kernel}

Definitions like these are compiled down to recursors:

\begin{lstlisting}
definition tail {A : Type} :
  Π{n}, vector A (succ n) → vector A n
| tail (h :: t) := t

definition zip {A B : Type} :
  Π{n}, vector A n → vector B n → vector (A × B) n
| zip nil nil         := nil
| zip (a::va) (b::vb) := (a, b) :: zip va vb

definition diag : Π{n}, vector (vector A n) n → vector A n
| diag nil             := nil
| diag ((a :: v) :: M) := a :: diag (map tail M)
\end{lstlisting}
}


\frame[containsverbatim]{\frametitle{Standard mode}

Specific to the standard mode:
\begin{itemize}
 \item \lstinline~Type.{0}~ (aka \lstinline{Prop}) is impredicative and (definitionally) proof irrelevant.
 \item quotient types (\lstinline{lift f H (quot.mk x) = f x} definitionally).
\end{itemize}
We use additional axioms for classical reasoning: propositional extensionality, Hilbert choice.

Lean keeps track of which definitions are computable.

\begin{lstlisting}
noncomputable definition inv_fun (f : X → Y)
  (s : set X) (dflt : X) (y : Y) : X :=
if H : ∃₀ x ∈ s, f x = y then some H else dflt

definition add (x y : ℝ) : ℝ := ...

noncomputable definition div (x y : ℝ) : ℝ :=
x * y⁻¹

\end{lstlisting}

}

\begin{frame}
\frametitle{HoTT mode}

In the HoTT mode:
\begin{itemize}
\item There is no \lstinline{Prop}.
 \item The Univalence Axiom is assumed.
 \item There are two primitive HITs in Lean: the $n$-truncation and \emph{quotients}.
  Given $A : \UU$ and $R : A \to A \to \UU$ the quotient is:\na
  \texttt{HIT} $quotient_A(R) :=$
  \begin{itemize}
  \item $q : A \to quotient_A(R)$
  \item $\fa{x, y : A}{R(x, y) \to q(x) = q(y)}$
  \end{itemize}
\end{itemize}

\bigskip

Many HITs can be constructed from these two:
  \begin{itemize}
    \item colimits;
    \item pushouts, hence also suspensions, spheres, joins, $\ldots$
    \item the propositional truncation;
    \item HITs with 2-constructors, such as the torus and Eilenberg-MacLane spaces $K(G,1)$;
    \item \mbox{}[WIP] $n$-truncations and certain ($\omega$-compact) localizations (Egbert Rijke, vD).
  \end{itemize}

\bigskip

\end{frame}


\begin{frame}{Lean's elaborator}

Fully detailed expressions in dependent type theory are long.

\bigskip

Systems of dependent type theory allow users to leave a lot of information implicit.

\bigskip

Such systems therefore:
\begin{itemize}
 \item Parse user input.
 \item Fill in the implicit information.
\end{itemize}
The last step is known as ``elaboration.''

\end{frame}


\begin{frame}{Lean's elaborator}

Lean has a powerful elaborator that handles:
\begin{itemize}
 \item implicit universe levels
 \item higher-order unification
 \item computational reductions
 \item ad-hoc overloading
 \item coercions
 \item type class inference
 \item tactic proofs
\end{itemize}

\end{frame}


%% \begin{frame}[fragile]{Higher-order unification}

%% \begin{lstlisting}
%% variables {A : Type} {B : A → Type} {C : Πa, B a → Type}
%% definition sigma_transport {a₁ a₂ : A} (p : a₁ = a₂)
%%   (x : Σ(b : B a₁), C a₁ b) : p ▸ x = ⟨p ▸ x.1, p ▸D x.2⟩ :=
%% by induction p; induction x; reflexivity
%% \end{lstlisting}\na
%% \lstinline{▸} is transport. \lstinline{.1} and \lstinline{.2} are projections.\nc
%% The first transport is along the family \lstinline{λ(a : A), Σ(b : B a), C a b}\nc
%% The second transport is along \lstinline{B}\nc
%% The last transport is a ``dependent transport'' which is a map \lstinline{C a₁ x.1 → C a₂ (p ▸ x.1)}

%% \end{frame}



\begin{frame}{Type classes}

Structures and type class inference have been optimized to handle the algebraic hierarchy.

\bigskip

The algebraic hierarchy consist of:
\begin{itemize}
 \item order structures (including lattices, complete lattices)
 \item additive and multiplicative semigroups, monoids, groups, \ldots
 \item rings, fields, ordered rings, ordered fields, \ldots
 \item modules over arbitrary rings, vector spaces, normed spaces, \ldots
 \item homomorphisms preserving appropriate parts of structures
\end{itemize}

\bigskip

In the HoTT library we also use type classes to infer truncatedness (\lstinline{is_trunc n A}).
%% Type classes work beautifully.

\end{frame}

\frame[containsverbatim]{\frametitle{Type classes}

\footnotesize

\begin{lstlisting}
structure has_mul [class] (A : Type) := (mul : A → A → A)

structure semigroup [class] (A : Type) extends has_mul A :=
  (is_set_carrier : is_set A)
  (mul_assoc : ∀a b c, mul (mul a b) c = mul a (mul b c))

structure monoid [class] (A : Type) extends semigroup A, has_one A :=
  (one_mul : ∀a, mul one a = a) (mul_one : ∀a, mul a one = a)

variables {A : Type} [monoid A]
definition pow (a : A) : ℕ → A
| 0     := 1
| (n+1) := pow n * a

theorem pow_add (a : A) (m : ℕ) : ∀n, a^(m + n) = a^m * a^n
| 0     := by rewrite [nat.add_zero, pow_zero, mul_one]
| (n+1) := by rewrite [add_succ, *pow_succ, pow_add, mul.assoc]

definition int.linear_ordered_comm_ring [instance] :
  linear_ordered_comm_ring int := ...
\end{lstlisting}
}

%% \frame[containsverbatim]{\frametitle{Type classes}

%% Type classes are also used in the standard library:
%% \begin{itemize}
%%  \item to infer straightforward facts (\lstinline{finite s}, \lstinline{is_subgroup G})
%%  \item to simulate classical reasoning constructively (\lstinline{decidable p})
%% \end{itemize}

%% \bigskip

%% They are used in the HoTT library:
%% \begin{itemize}
%%  \item for category theory
%%  \item to infer truncatedness (\lstinline{is_trunc n A})
%%  \item to infer half-adjoint equivalence (\lstinline{is_equiv f})
%% \end{itemize}

%% }

\frame[containsverbatim]{\frametitle{Calculational proofs}

$$\sum_{(a, b) : \Sigma_{a : A}B(a)}C(a) \simeq \sum_{(a, c) : \Sigma_{a : A}C(a)}B(a)$$

\small

\begin{lstlisting}[gobble=2]
  definition sigma_assoc_comm_equiv {A : Type} (B C : A → Type)
    : (Σ(v : Σa, B a), C v.1) ≃ (Σ(u : Σa, C a), B u.1) :=
  calc
    (Σ(v : Σa, B a), C v.1)
        ≃ (Σa (b : B a), C a)     : sigma_assoc_equiv
    ... ≃ (Σa (c : C a), B a)     : sigma_equiv_sigma_right
                                      (λa, !comm_equiv_nondep)
    ... ≃ (Σ(u : Σa, C a), B u.1) : sigma_assoc_equiv

\end{lstlisting}

}

\frame[containsverbatim]{\frametitle{Tactics and terms}

In Lean, we can present a proof as a term, as in Agda, with nice syntactic sugar.

\bigskip

We can also use tactics.

\bigskip

The two modes can be mixed freely:
\begin{itemize}
 \item anywhere a term is expected, \lstinline{begin ... end} or \lstinline{by} enter tactic mode.
 \item within tactic mode, \lstinline{have ..., from ...} or \lstinline{show ..., from ...} or \lstinline{exact} specify terms.
\end{itemize}

}

\frame[containsverbatim]{\frametitle{Example term proof}

\scriptsize

\begin{lstlisting}
theorem sqrt_two_irrational {a b : ℕ} (co : coprime a b) :
  a^2 ≠ 2 * b^2 :=
assume H : a^2 = 2 * b^2,
have even (a^2),
  from even_of_exists (exists.intro _ H),
have even a,
  from even_of_even_pow this,
obtain (c : ℕ) (aeq : a = 2 * c),
  from exists_of_even this,
have 2 * (2 * c^2) = 2 * b^2,
  by rewrite [-H, aeq, *pow_two, mul.assoc, mul.left_comm c],
have 2 * c^2 = b^2,
  from eq_of_mul_eq_mul_left dec_trivial this,
have even (b^2),
  from even_of_exists (exists.intro _ (eq.symm this)),
have even b,
  from even_of_even_pow this,
assert 2 ∣ gcd a b,
  from dvd_gcd (dvd_of_even `even a`) (dvd_of_even `even b`),
have 2 ∣ 1,
  by rewrite [gcd_eq_one_of_coprime co at this]; exact this,
show false,
  from absurd `2 ∣ 1` dec_trivial
\end{lstlisting}
}

\frame[containsverbatim]{\frametitle{Example tactic proof}

\scriptsize

\begin{lstlisting}[gobble=2]
  variable (P : S¹ → Type)
  definition circle.rec_unc (v : Σ(p : P base), p =[loop] p)
    : Π(x : S¹), P x :=
  begin
    intro x, induction v with p q, induction x,
    { exact p},
    { exact q}
  end

  definition circle_pi_equiv
    : (Π(x : S¹), P x) ≃ Σ(p : P base), p =[loop] p :=
  begin
    fapply equiv.MK,
    { intro f, exact ⟨f base, apd f loop⟩},
    { exact circle.rec_unc P},
    { intro v, induction v with p q, fapply sigma_eq,
      { reflexivity},
      { esimp, apply pathover_idp_of_eq, apply rec_loop}},
    { intro f, apply eq_of_homotopy, intro x, induction x,
      { reflexivity},
      { apply eq_pathover_dep, apply hdeg_squareover, esimp, apply rec_loop}}
  end
\end{lstlisting}
}

\begin{frame}{Current plans}

Leo is now doing a major rewrite of the system.
\begin{itemize}
\item The elaborator will be slightly less general, but much more stable and predictable.
\item ``Let'' definitions will be added to the kernel.
\item There will be a better foundation for automation (e.g. goals with indexed hypotheses).
\item There will be a byte-code compiler and fast evaluator for Lean.\\
 This makes it possible to use Lean as an interpreted programming language.
\item Monadic interfaces will make it possible to write custom tactics from within Lean.
\item There will be powerful automation; a general theorem prover, a term simplifier and a flexible framework for adding domain specific tools.
\end{itemize}

\end{frame}

%% \begin{frame}{Automation}

%% Plans for automation:
%% \begin{itemize}
%%  \item A general theorem prover and term simplifier, with
%%  \begin{itemize}
%%  \item awareness of type classes
%%  \item powerful unification and e-matching
%%  \end{itemize}
%%  \item Custom methods for arithmetic reasoning, linear and nonlinear inequalities.
%%  \item A flexible framework for adding domain specific tools.
%% \end{itemize}

%% \end{frame}

\begin{frame}{Standard library}

Already has:
\begin{itemize}
 \item datatypes: booleans, lists, tuples, finsets, sets
 \item number systems: nat, int, rat, real, complex
 \item the algebraic hierarchy, through ordered fields
 \item ``big operations'': finite sums and products, etc.
 \item elementary number theory (e.g.~primes, gcd's, unique factorization, etc.)
 \item elementary set theory
 \item elementary group theory
 \item beginnings of analysis: topological spaces, limits, continuity, the intermediate value theorem
\end{itemize}

\end{frame}

%------------------------------------------------

%% \begin{frame}
%%   \begin{center}
%%    {\huge Demo}\nn \nn

%%    Browser version available at:\na
%%    \url{leanprover.github.io/tutorial/?live&hott}
%%   \end{center}
%% \end{frame}

%% Language                     files          blank        comment           code
%% -------------------------------------------------------------------------------
%% Lean                           160           7214           1393          28227
%% Coq                            281           6204           2752          31421
%% Agda                           162           4468            562          17799
%% UniMath                        142          13790           2806          52091
%% Note: in comment column only single-line comments are counted

\begin{frame}[fragile]{HoTT library}

  \makebox[\textwidth][c]{{\small \formalizedtable}}\nb
          {\footnotesize this table is online: \url{github.com/leanprover/lean/blob/master/hott/book.md}}

\end{frame}
%------------------------------------------------

%% \begin{frame}
%% \frametitle{Statistics}
%% HoTT-Lean: $\sim$28k LOC\na
%% HoTT-Coq: $\sim$31k LOC (in \texttt{theories/} folder)\na
%% HoTT-Agda: $\sim$18k LOC (excluding \texttt{old/} folder)\na
%% UniMath: $\sim$52k LOC\nn
%% \footnotesize(excluding blank lines and single-line comments)
%% \end{frame}

%------------------------------------------------

\begin{frame}[fragile]
\frametitle{Lean-HoTT library}
The library also includes:
\begin{itemize}
\item A library of squares, cubes, pathovers, squareovers, cubeovers (based on the paper by Licata and Brunerie)
\begin{lstlisting}
definition circle.rec {P : S¹ → Type}
  (Pbase : P base) (Ploop : Pbase =[loop] Pbase)
  (x : S¹) : P x
\end{lstlisting}
\item<2-> A library of pointed types, pointed maps, pointed homotopies, pointed equivalences
\begin{lstlisting}[gobble=2]
  definition loopn_ptrunc_pequiv
    (n : ℕ₋₂) (k : ℕ) (A : Type*) :
    Ω[k] (ptrunc (n+k) A) ≃* ptrunc n (Ω[k] A)
\end{lstlisting}

\item<3-> Category theory, e.g. limits, colimits and exponential laws:
\begin{lstlisting}
definition functor_functor_iso (C D E : Precategory) :
  (C ^c D) ^c E ≅c C ^c (E ×c D)
\end{lstlisting}
%\item<4-> The h-space structure of $S^3$.
\end{itemize}
\end{frame}

%------------------------------------------------

\begin{frame}[fragile]
\frametitle{Homotopy Theory in Lean}
The Hopf fibration (by Ulrik Buchholtz):
\begin{lstlisting}[gobble=2]
  variables (A : Type) [H : h_space A] [K : is_conn 0 A]

  definition hopf : susp A → Type :=
  susp.elim_type A A
    (λa, equiv.mk (λx, a * x) !is_equiv_mul_left)

  definition hopf.total (A : Type) [H : h_space A]
    [K : is_conn 0 A] : sigma (hopf A) ≃ join A A

  definition circle_h_space : h_space S¹
  definition sphere_three_h_space : h_space (S 3)
\end{lstlisting}

\end{frame}

%------------------------------------------------

%% \begin{frame}[fragile]
%%   \frametitle{Homotopy Theory in Lean}
%%   We have ported formalizations from Agda, such as the Freudenthal equivalence
%% \begin{lstlisting}
%% definition freudenthal_pequiv (A : Type*) {n k : ℕ}
%%   [is_conn n A] (H : k ≤ 2 * n) :
%%   ptrunc k A ≃* ptrunc k (Ω (psusp A))
%% \end{lstlisting}
%% and the associativity of join (by Jakob von Raumer)
%% \begin{lstlisting}[gobble=2]
%%   definition join.assoc (A B C : Type) :
%%     join (join A B) C ≃ join A (join B C)
%% \end{lstlisting}


%% \end{frame}

%------------------------------------------------

%% \begin{frame}[fragile]
%% \frametitle{Homotopy Theory in Lean}
%% Truncated Whitehead's principle:
%% \begin{lstlisting}[gobble=2]
%%   definition whitehead_principle (n : ℕ₋₂) {A B : Type}
%%     [HA : is_trunc n A] [HB : is_trunc n B] (f : A → B)
%%     (H' : is_equiv (trunc_functor 0 f))
%%     (H : Πa k, is_equiv
%%                  (π→*[k + 1] (pmap_of_map f a))) :
%%     is_equiv f
%% \end{lstlisting}

%% Here `\lstinline{pmap_of_map f a}' is the pointed map $(A,a) \stackrel f\to (B, f(a))$.

%% \end{frame}

%------------------------------------------------

\begin{frame}[fragile]
\frametitle{Homotopy Theory in Lean}
Eilenberg-MacLane spaces (based on the paper by Licata and Finster):
\begin{lstlisting}[gobble=2]
  definition EM : CommGroup → ℕ → Type*
  variables (G : CommGroup) (n : ℕ)
  definition homotopy_group_EM :
    πg[n+1] (EM G (n+1)) ≃g G
  theorem is_conn_EM : is_conn (n.-1) (EM G n)
  theorem is_trunc_EM : is_trunc n (EM G n)

  definition EM_pequiv_1 {X : Type*} (e : π₁ X ≃g G)
    [is_conn 0 X] [is_trunc 1 X] : EM G 1 ≃* X
  -- TODO for n > 1
\end{lstlisting}


\end{frame}

%------------------------------------------------

%% \begin{frame}[fragile]
%% \frametitle{Homotopy Theory in Lean}
%% Seifert-van Kampen Theorem: the fundamental groupoid of a pushout is the pushout of the fundamental groupoids.

%% \end{frame}

%------------------------------------------------

%% \begin{frame}
%% \frametitle{LES of homotopy groups}
%% The long exact sequence of homotopy groups.\nn
%% Given a pointed map $f : X \to Y$.\nn
%% \visible<2->{Define $F :\equiv \text{fiber}_f(y_0) :\equiv (\ex{x : X}{f(x)=y_0})$.}\nn\nn
%% \makebox[\textwidth]{%
%% \begin{tikzpicture}[>=stealth',auto,node distance=3cm,
%%     thick,main node/.style={font=\sffamily\Large\bfseries},text height=1.5ex, node distance=7mm]
%%   \node (Y)     at (0,0) {$Y$};
%%   \node[left = of Y] (X) {$X$};
%%   \node[left = of X, visible on=<2->] (F) {$F$};
%%   \node[left = of F, visible on=<3->] (F2) {$F^{(2)}$};
%%   \node[left = of F2, visible on=<3->] (F3) {$F^{(3)}$};
%%   \node[left = of F3, visible on=<3->] (F4) {$F^{(4)}$};
%%   \node[left = of F4, visible on=<3->] (F5) {$F^{(5)}$};
%%   \node[left = of F5, visible on=<3->] (F6) {$F^{(6)}$};
%%   \node[below = of F2, visible on=<4->] (OY) {$\Omega Y$};
%%   \node[below = of F3, visible on=<4->] (OX) {$\Omega X$};
%%   \node[below = of F4, visible on=<4->] (OF) {$\Omega F$};
%%   \node[below = of F5, visible on=<4->] (OF2) {$\Omega F^{(2)}$};
%%   \node[below = of F6, visible on=<4->] (OF3) {$\Omega F^{(3)}$};
%%   \node[below = of OF2, visible on=<6->] (O2Y) {$\Omega^2 Y$};
%%   \node[below = of OF3, visible on=<6->] (O2X) {$\Omega^2 X$};
%%   \node[rotate=90, visible on=<4->] (foo) at ($(F2)!0.5!(OY)$) {$\simeq$};
%%   \node[rotate=90, visible on=<4->] (foo) at ($(F3)!0.5!(OX)$) {$\simeq$};
%%   \node[rotate=90, visible on=<4->] (foo) at ($(F4)!0.5!(OF)$) {$\simeq$};
%%   \node[rotate=90, visible on=<4->] (foo) at ($(F5)!0.5!(OF2)$) {$\simeq$};
%%   \node[rotate=90, visible on=<4->] (foo) at ($(F6)!0.5!(OF3)$) {$\simeq$};
%%   \node[rotate=90, visible on=<6->] (foo) at ($(OF2)!0.5!(O2Y)$) {$\simeq$};
%%   \node[rotate=90, visible on=<6->] (foo) at ($(OF3)!0.5!(O2X)$) {$\simeq$};
%%   \path[every node/.style={font=\sffamily\small}]
%%   (X) edge[->] node [above] (f){$f$} (Y)
%%   (F) edge[->, visible on=<2->] node [above] (f){$p_1$} (X)
%%   (F2) edge[->, visible on=<3->] node [above] (f){$p_1^{(2)}$} (F)
%%   (F3) edge[->, visible on=<3->] node [above] (f){$p_1^{(3)}$} (F2)
%%   (F4) edge[->, visible on=<3->] node [above] (f){$p_1^{(4)}$} (F3)
%%   (F5) edge[->, visible on=<3->] node [above] (f){$p_1^{(5)}$} (F4)
%%   (F6) edge[->, visible on=<3->] node [above] (f){$p_1^{(6)}$} (F5)
%%   (OX) edge[->, visible on=<5->] node [above] (f){$-\Omega f$} (OY)
%%   (OF) edge[->, visible on=<5->] node [above] (f){$-\Omega p_1$} (OX)
%%   (OF2) edge[->, visible on=<5->] node [above] (f){$-\Omega p_1^{(2)}$} (OF)
%%   (OF3) edge[->, visible on=<5->] node [above] (f){$-\Omega p_1^{(3)}$} (OF2)
%%   (O2X) edge[->, visible on=<6->] node [above] (f){$\Omega^2 f$} (O2Y)
%%   (OY) edge[->, visible on=<7->] node [below right] (f){$\delta$} (F)
%%   (O2Y) edge[->, visible on=<7->] node [below right] (f){$-\Omega\delta$} (OF);
%% \end{tikzpicture}%
%% }

%% \end{frame}

%% %------------------------------------------------

%% \begin{frame}
%% \frametitle{LES of homotopy groups}
%%   \begin{columns}[c]
%%     \begin{column}{.5\textwidth}
%%       \makebox[\textwidth][c]{
%% \begin{tikzpicture}[>=stealth',auto,node distance=3cm,
%%     thick,main node/.style={font=\sffamily\Large\bfseries},text height=1.5ex, node distance=12mm]
%%   \node (Y)     at (0,0) {$\pi_0(Y)$};
%%   \node[left = of Y] (X) {$\pi_0(X)$};
%%   \node[left = of X] (F) {$\pi_0(F)$};
%%   \node[above = of Y] (OY) {$\pi_1(Y)$};
%%   \node[above = of X] (OX) {$\pi_1(X)$};
%%   \node[above = of F] (OF) {$\pi_1(F)$};
%%   \node[above = of OY] (O2Y) {$\pi_2(Y)$};
%%   \node[above = of OX] (O2X) {$\pi_2(X)$};
%%   \node[above = of OF] (O2F) {$\pi_2(F)$};
%%   \path[every node/.style={font=\sffamily\small}]
%%   (X) edge[->] node [above] (f){$\pi_0(f)$} (Y)
%%   (F) edge[->] node [below] (f){$\pi_0(p_1)$} (X)
%%   (OY) edge[->] node [left] (f){$\pi_0(\delta)\quad\mbox{}$} (F)
%%   (OX) edge[->] node [above] (f){$\only<2>{\fbox{$-$}}\only<1>{-}\pi_1(f)$} (OY)
%%   (OF) edge[->] node [below] (f){$\only<2>{\fbox{$-$}}\only<1>{-}\pi_1(p_1)$} (OX)
%%   (O2Y) edge[->] node [left] (f){$\only<2>{\fbox{$-$}}\only<1>{-}\pi_1(\delta)\quad\mbox{}$} (OF)
%%   (O2X) edge[->] node [above] (f){$\pi_2(f)$} (O2Y)
%%   (O2F) edge[->] node [below] (f){$\pi_2(p_1)$} (O2X);
%% \end{tikzpicture}}
%%     \end{column}
%%     \begin{column}{.45\textwidth}%
%%       \visible<4->{%
%%       How do we formulate this?\nn
%%       The obvious thing is to have a sequence $Z : \N \to \UU$ and maps $f_n : Z_{n+1}\to Z_n$.\nn
%%       \textbf{Problem}: $Z_{3n}=\pi_n(Y)$ doesn't hold definitionally, hence $f_{3n}=\pi_n(f)$ isn't even well-typed.
%%       }
%%     \end{column}
%%   \end{columns}

%% \end{frame}

%% %------------------------------------------------

%% \begin{frame}[fragile]
%% \frametitle{LES of homotopy groups}
%% \textbf{Better}: Take $Z : \N \times 3 \to \UU$, we can define $Z$ by
%% \begin{align*}Z_{(n,0)}=\pi_n(Y)&&Z_{(n,1)}=\pi_n(X)&&Z_{(n,2)}=\pi_n(F).\end{align*}
%% Then we can define the maps $f_x : Z_{succ(x)}\to Z_x$, where $succ$ is the successor function for $\N\times3$.\nn
%% \visible<2->{We define chain complexes over an arbitrary type with a successor operation.}

%% \end{frame}

%% %------------------------------------------------

%% \begin{frame}[fragile]
%% \frametitle{LES of homotopy groups}
%% \begin{lstlisting}[gobble=2]
%%   definition homotopy_groups : +3ℕ → Set*
%%   | (n, fin.mk 0 H) := π*[n] Y
%%   | (n, fin.mk 1 H) := π*[n] X
%%   | (n, fin.mk k H) := π*[n] (pfiber f)

%%   definition homotopy_groups_fun : Π(n : +3ℕ),
%%     homotopy_groups (S n) →* homotopy_groups n
%%   | (n, fin.mk 0 H) := π→*[n] f
%%   | (n, fin.mk 1 H) := π→*[n] (ppoint f)
%%   | (n, fin.mk 2 H) := π→*[n] boundary_map ∘*
%%       pcast (ap (ptrunc 0) (loop_space_succ_eq_in Y n))
%%   | (n, fin.mk (k+3) H) := begin exfalso,
%%       apply lt_le_antisymm H, apply le_add_left end
%% \end{lstlisting}
%% \end{frame}

%% %------------------------------------------------

%% \begin{frame}[fragile]
%% \frametitle{LES of homotopy groups}
%% Then we prove:
%% \begin{itemize}
%% \item These maps form a chain complex
%% \item This chain complex is exact
%% \item \lstinline{homotopy_groups (n + 2, k)} are commutative groups.
%% \item \lstinline{homotopy_groups (1, k)} are groups.
%% \item \lstinline{homotopy_groups_fun (n + 1, k)} are group homomorphisms.
%% \end{itemize}
%% \end{frame}

%------------------------------------------------

\begin{frame}
  \frametitle{LES of homotopy groups}
  Given a pointed map $f : X \to Y$. Then the following is a long exact sequence:
  {\footnotesize (the image of any map is exactly the kernel of the next map)}\nn
      \makebox[\textwidth][c]{
\begin{tikzpicture}[>=stealth',auto,node distance=3cm,
    thick,main node/.style={font=\sffamily\Large\bfseries},text height=1.5ex, node distance=12mm]
  \node (Y)     at (0,0) {$\pi_0(Y)$};
  \node[left = of Y] (X) {$\pi_0(X)$};
  \node[left = of X] (F) {$\pi_0(F)$};
  \node[above = of Y] (OY) {$\pi_1(Y)$};
  \node[above = of X] (OX) {$\pi_1(X)$};
  \node[above = of F] (OF) {$\pi_1(F)$};
  \node[above = of OY] (O2Y) {$\pi_2(Y)$};
  \node[above = of OX] (O2X) {$\pi_2(X)$};
  \node[above = of OF] (O2F) {$\pi_2(F)$};
  \node[above = 4mm of O2X] (O3X) {$\vdots$};
  \path[every node/.style={font=\sffamily\small}]
  (X) edge[->] node [above] (f){$\pi_0(f)$} (Y)
  (F) edge[->] node [below] (f){$\pi_0(p_1)$} (X)
  (OY) edge[->] node [left] (f){$\pi_0(\delta)\quad\mbox{}$} (F)
  (OX) edge[->] node [above] (f){$\pi_1(f)$} (OY)
  (OF) edge[->] node [below] (f){$\pi_1(p_1)$} (OX)
  (O2Y) edge[->] node [left] (f){$\pi_1(\delta)\quad\mbox{}$} (OF)
  (O2X) edge[->] node [above] (f){$\pi_2(f)$} (O2Y)
  (O2F) edge[->] node [below] (f){$\pi_2(p_1)$} (O2X);
\end{tikzpicture}}\nb
      $F$ is the fiber of $f$ and $p_1$ is the first projection.

\end{frame}

%------------------------------------------------

\begin{frame}[fragile]
\frametitle{LES of homotopy groups}

Corollaries:
\begin{lstlisting}[gobble=2]
  theorem is_equiv_π_of_is_connected {A B : Type*}
    {n k : ℕ} (f : A →* B) [H : is_conn_fun n f]
    (H2 : k ≤ n) : is_equiv (π→[k] f)
\end{lstlisting}

Combined with Hopf fibration:

\begin{lstlisting}[gobble=2]
  definition π2S2 : πg[1+1] (S. 2) ≃g gℤ
  definition πnS3_eq_πnS2 (n : ℕ) :
    πg[n+2 +1] (S. 3) ≃g πg[n+2 +1] (S. 2)
\end{lstlisting}

Combined with Freudenthal Suspension Theorem:

\begin{lstlisting}[gobble=2]
  definition πnSn (n : ℕ) : πg[n+1] (S. (succ n)) ≃g gℤ
  definition π3S2 : πg[2+1] (S. 2) ≃g gℤ
\end{lstlisting}

\end{frame}

%------------------------------------------------

\begin{frame}[fragile]
\frametitle{Conclusion}
\begin{itemize}
  \item Lean is an exciting new proof assistant.
  \item The Lean-HoTT library is quite big and growing quickly.
  \item The Lean-HoTT library contains a good basis for serious formalizations.
  \item There is currently an ongoing collaboration to formalize the Serre Spectral Sequence.
    %(j.w.w. Ulrik Buchholtz, Mike Shulman, Jeremy Avigad, Steve Awodey, Egbert Rijke, Clive Newstead)
\end{itemize}
\end{frame}

%------------------------------------------------

\begin{frame}
  \begin{center}
   {\huge Thank you}
  \end{center}
\end{frame}

%------------------------------------------------

\end{document}
